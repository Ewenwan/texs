\chapter{实验总结}
通过完成本次实验,熟悉了数字电路全定制流程,包括最初的功能分析、电路图设计、功能验证、时序分析与优化以及后来的版图设计以及DRC、LVS验证等。实验结果表明,设计的NORM指令模块可以实现正确的功能(NC、HSpice结果),输入至输出的响应延时为395ps(HSpice结果),且版图设计与电路图设计一致、没有DRC错误(通过DRC、LVS验证)。下面对本次实验进行总结。
\section{试验中遇到的问题与解决办法}
回顾整个实验流程,遇到的几个主要问题及其解决办法如下:
\begin{itemize}
\item \textbf{由Composer导出电路图的NC-Verilog文件以及.cdl网表过程中,导出失败} \\
\textbf{解决办法:}通过对导出文件的过程进行分析,发现在导出顶层模块的上述文件时,需要首先将顶层模块包含的所有底层模块的上述文件导出,然后即可解决错误。
\item \textbf{在基于设计的.cdl网表进行时序分析时,输出并不会跟随输入变化}\\
\textbf{解决办法:}通过.sp文件中瞬态源的周期得以解决,即将在Src<27>输入端口的脉冲源的周期由最开始的2ns增加为4ns,同时将高电平维持时间由原来的1ns改为2ns后,输出正常,其结果见图\ref{fig4.1}。
\item \textbf{版图设计中DRC验证出现问题} \\
\textbf{解决办法:}一般发生DRC违反,主要由版图设计中存在电器规则违反所致,如金属层之间的间距过小、金属的面积过小等错误。根据提示修改这些错误后,所有模块均通过DRC检查。
\item \textbf{版图设计中LVS验证出现问题}\\
\textbf{解决办法:}一般发生LVS错误,可以是.cdl文件与.gds文件之间存在不一致导致,包括版图中器件的尺寸与电路图中器件的尺寸或名字不一致、连线存在错误等。经过对.cdl文件以及版图中不一致的地方进行修改后,所有模块的LVS检查均通过验证。
\end{itemize}
\section{实验收获与不足}
从这一次实验中熟悉了各种全定制工具的使用,对于Virtuoso工具的使用更加熟练,对于版图设计规则有了更全的认识,能够较好的运用工具来实现设计,对全定制设计有了更全面的理解。

通过完成tms320 DSP的NORM指令的全定制,对实际数字电路功能的实现有了更深的理解,包括功能分析以及具体电路的设计等,为以后的工作打下了坚实的基础。在电路设计过程中,学习了如何通过阅读论文来找出比较有效的解决方案,更重要的,通过完成低功耗的LZC模块,了解了一种低功耗设计思路,即通过逻辑优化以及器件共享来减少所需的器件的数量,从而降低电路的整体功耗。

在这段时间里部分完成NORM指令的全定制设计,自己动手实践能力有了很大的提高,通过跟同学,老师的交流,体会到合作的重要性。更重要的,通过解决遇到的问题,锻炼了解决问题的能力。

当然本次实验还存在许多不足,包括最终的版图设计中对电路的面积还存在优化的空间,布局可以更合理一些。在前期功能验证中,下一步的工作可以是引入黄金模型对电路图的正确性进行验证,虽然从已有的几个输入激励中全部得到正确的结果,但还需要提高验证的覆盖率。

