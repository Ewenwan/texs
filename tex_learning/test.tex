\documentclass[a4paper, 11pt]{article}

\author{smh \and smher}

\title{\textsl{first try}}

\usepackage{amsfonts}
\usepackage{amsmath}
\usepackage{amsthm}
\usepackage{amsbsy}


\begin{document}

\maketitle
\tableofcontents

\part{Introduce}
\section{Text Edit}
I read Kunth divides the people working with \TeX{} into \TeX{} nicians and \TeX perts. \\
Today is \today. \\

you can \textsl{lean} on me !\\

\ldots when Einstein introduced his formula 
\begin{equation}
e = m \cdot c^2 \;\footnote{famous equation} ,
\end{equation}
which is at the same time the most widely known and the least well understood physical formula.

\mbox{123 345 789}\\
\fbox{123 456 789}\\

``please 'u' akdfj''

hello-boys\\
pages 34--35\\
yes---or no?\\
$0$, $1$, and $-1$\\
www.baidu.com/\~{}bush\\
www.baidu.com/$\sim$demo\\
it's $-30\,^{\circ}\mathrm{C}$\\
not like this ... but like this: \\new york, tokyo, budapest, \ldots \\
Not shelfful\\
but shelf\mbox{}ful\\

H\^otel, na\"ive, \'el\`eve,\\
sm\o rrebr\o d, !`Se\~norita!,\\
Sch\"onbrunner Schlo\ss Stra\ss e, \c o\\

%\section{first}
%	the first section.
%\subsection{second}
%	the second section.\label{sec:that}
%\section*{first-s}
%	first second
%\subsection*{second-s}
%	second second

\paragraph{para-first}
	first paragraph
\subparagraph{sub-para-first}
	first subparagraph\\

A reference to this subsection \label{sec:this} looks like :
``see section~\ref{sec:that} on page ~\pageref{sec:this}.''
Footnote\footnote{this is a footnote .} are often used by people using \LaTeX.
\underline{hello \LaTeX}.\\
\emph{hello world}\\
\emph{if you use emphasizing inside a piece of emphasized text, then \LaTeX{} uses the \emph{normal} font for more emphasizing.}\\
\textit{you can also \emph{emphasize} text if it is set in italics, }\textsf{in a \emph{sans-serif} font, } \texttt{or in \emph{typewrite} style.} \\

%\part{environment}
%\section{Environment}
\subsection{list}
\flushleft
\begin{enumerate}
\item You can mix the list envrionments to your taste:
\begin{itemize}
\item But it might start to look silly.
\item[-] With a dash.
\end{itemize}
\item Therefore remember:
\begin{description}
\item[Stupid] things will not become smart because they are in a list.
\item[Smart] things, though can be presented beautifully in a list.
\end{description}
\end{enumerate}

\subsection{align}
\begin{flushleft}
This text is \\ left-alighed.
\LaTeX{} is not trying to make each line the same length.
\end{flushleft}
\begin{flushright}
This is text right-\\alighed.
\LaTeX{} is not trying to make each line the same length.
\end{flushright}
\begin{center}
At the center \\ of the earth.
\end{center}

\subsection{quote}
A typographical rule of thumb for the line length is :
\begin{quote}
On average, no line should be longer than 66 characters.
\end{quote}
This is why \LaTeX{} pages have such large borders by default and also why multicolumn print is used in newspapers.
\begin{flushleft}
I know only one English poem by heart. It is about Humpty Dumpty.
\begin{verse}
Humpty Dumpty sat on a wall:\\
Humpty Dumpty had a great fall.\\
All the king's horses and the king's men\\
Couldn't put Humpty together again.
\end{verse}
\end{flushleft}

\subsection{abstract}
\begin{abstract}
The abstract abstract.
\end{abstract}

\subsection{print}
the \verb|\ldots| command \ldots 
\begin{verbatim}
10  print "hello world";
20  Goto 10
\end{verbatim}

\subsection{table}
\begin{tabular}{|r|l|}
\hline
7C0 & hexadecimal\\
3700 & octal \\ \cline{2-2}
11111000000 & binary \\
\hline
\hline
1984 & decimal \\
\hline
\end{tabular}\label{sec:tab1}
\\
table~\ref{sec:tab1} is the first table\footnote{this is the first table.}. 
\\
\begin{tabular}{c r @{.} l}
Pi expression & \multicolumn{2}{c}{Value} \\
\hline
$\pi$ & $3$ & 1416\\
$\pi$ & 36 & 46\\
$(\pi^{\pi})^{\pi}$ & 80062 & 7\\
\end{tabular}\label{sec:tab2}\\
%\caption{first table}\\
\ref{sec:tab2} is the second table\footnote{this is the second table}.\\
\begin{tabular}{|c|c|}
\hline
\multicolumn{2}{|c|}{Ene}\\
\hline
Mene & Muh!\\
\hline
\end{tabular}

\subsection{floating bodies}
figure~\ref{white} is an example of Pop-Art.
\begin{figure}[!hbp]
\makebox[\textwidth]{\framebox[5cm]{\rule{0pt}{5cm}}}
\caption{Five by Five in Centimeters.\label{white}}
\end{figure}

\subsection{protect}
\verb|\footnote| and \verb|\phantom| is fragile commands and should be protected.\\
\begin{enumerate}
\item Usage
\begin{itemize}
\item[-] \verb|\protect|
\item[-] \verb|\protect|\verb|\footnote{and protect my footnotes}|\\
\end{itemize}
\item Description
\end{enumerate}

\part{MATH}
\section{math}
\subsection{Intro.}
Add $a$ squared and $b$ squared to get $c$ squared. Or, using a more mathematical approach:$c^2 = a^2 + b^2$.\\

\TeX{} is pronounced as \(\tau\epsilon\chi\). \\[6pt]
100~m$^{3}$ of water\\[6pt]
This comes from my 
\begin{math}
\heartsuit
\end{math}\\

Add $a$ squared and $b$ squared to get get $c$ squared. Or, using a more mathematical approach:
\begin{displaymath}
c^{2} = a^{2} + b^{2}
\end{displaymath}
or you can type less with :
\[a^{2} + b^{2} = c^{2}\] 

\begin{equation}\label{eq:eps}
\epsilon > 0
\end{equation}
From(\ref{eq:eps}), we gather \ldots{} From (\ref{eq:eps}) we do the same.\\
Two type display : first, $lim_{n \to \infty}\sum_{k=1}^{n}\frac{1}{k^2} = \frac{\pi ^{2}}{6}$; second,\begin{displaymath}
lim_{n \to \infty}\sum_{k=1}^{n}\frac{1}{k^2} = \frac{\pi ^{2}}{6}
\end{displaymath}

\begin{equation}
\forall x in \mathbf{R}:
\qquad x^2 \geq 0 
\end{equation}

\begin{equation}
x^2 \geq 0 \qquad
\textrm{for all }x \in \mathbf{R} 
\end{equation}

\begin{displaymath}
x^2 \geq 0 \qquad
\textrm{for all }x \in \mathbb{R}
\end{displaymath}

\subsection{array}
\begin{equation}
a^x + y \neq a^{x+y} 
\end{equation}

\subsection{basic elements}
$\alpha, \xi, \pi, \mu, \Phi, \Omega $ \\
$a_{1}$ \qquad $x^{2} \qquad$ $e^{-\alpha t} \qquad a_{ij}^{3}$\\
$e^{x^2} \neq {e^x}^{2}$ \\

\(\sqrt{x}\) \qquad \(\sqrt{x^{2} + \sqrt{y}}\) \\
$\sqrt[3]{n}$ \\ [3pt]
$\surd[x^2 + y^2]$\\

$\overline{m+n}$ \qquad $\underline{m+n}$ \\
\begin{displaymath}
\underbrace{a + b + \cdots +z}_{26}=100
\end{displaymath}
\begin{displaymath}
y=x^2 \qquad y' = 2x\qquad y''=2
\end{displaymath}
\[\vec{a}\qquad\overrightarrow{AB}\qquad\overleftarrow{AB}\]

\begin{displaymath}
v = {\sigma}_1 \cdot {\sigma}_2 \cdot {\tau}_1 \cdot {\tau}_2
\end{displaymath}

\[\lim_{x \to 0}\frac{\sin x}{x} = 1\]
\[\lim_{x \rightarrow 0}\frac{\sin x}{x} = 1\]

$a\bmod b $\\
$a \equiv a \pmod{b}$\\

$1\frac{1}{2}$~hours
\begin{displaymath}
\frac{x^2}{k+1}\qquad x^{\frac{2}{k+1}}\qquad x^{1/2}
\end{displaymath}

\begin{displaymath}
\binom{n}{k} \qquad \mathrm{C}_n^k
\end{displaymath}

\begin{displaymath}
\int f_N(x) \stackrel{!}{=} 1
\end{displaymath}

\begin{displaymath}
\sum_{i=1}^{n}\qquad \int_{0}^{\frac{\pi}{2}}\qquad \prod_{\epsilon}
\end{displaymath}

\begin{displaymath}
\sum_{\substack{0<i<n \\ 1<j<m}}P(i,j) = \sum_{\begin{subarray}{l}
i \in I \\ 1<j<m
\end{subarray}}Q(i,j)
\end{displaymath}

\begin{displaymath}
{a,b,c} \neq \{a, b, c\}
\end{displaymath}

\begin{displaymath}
1 + \left(\frac{1}{1- x^2}\right)^3
\end{displaymath}

\begin{displaymath}
1 + (\frac{1}{1- x^2})^3
\end{displaymath}

$\big((x+1)(x-1)\big)^2$\\
$\big(\Big(\bigg(\Bigg($ \quad $\big\}\Big\}\bigg\}\Bigg\}$
\quad
$\big\| \Big\| \bigg\| \Bigg\|$\\

\begin{displaymath}
x_1, x_2, \ldots, x_n \qquad
x_1, x_2, \cdots, x_n
\end{displaymath}

\subsection{math black}
\newcommand{\ud}{\mathrm{d}}
\begin{displaymath}
\int\!\!\!\int_{D}g(x,y)\,\ud x\, \ud y
\end{displaymath}
instead of
\begin{displaymath}
\int\int_{D} g(x,y)\ud x\ud y
\end{displaymath}
\begin{displaymath}
\iint_{D} g(x,y) \, \ud x\, \ud y
\end{displaymath}

\subsection{verticle align}
\begin{displaymath}
\mathbf{X} = 
\left(\begin{array}{ccc}
x_{11} & x_{12} & \ldots\\
x_{21} & x_{22} & \ldots\\
x_{31} & x_{32} & \ldots\\
\vdots & \vdots & \ddots
\end{array}\right)
\end{displaymath}

\begin{displaymath}
y = \left\{\begin{array}{ll}
a & \textrm{if $d>c$}\\
b+x & \textrm{in the morining}\\
l & \textrm{all day long}
\end{array}\right.
\end{displaymath}

\begin{displaymath}
\left(\begin{array}{c|c}
1 & 2\\
\hline
3 & 4
\end{array}\right)
\end{displaymath}

\begin{eqnarray}
f(x) & = & \cos x \\
f'(x) & = & - \sin x \\
\int_{0}^{x} f(y) dy & = & \sin x  
\end{eqnarray}

{\setlength\arraycolsep{2pt}
\begin{eqnarray}
\sin x & = & x - \frac{x^3}{3!} + \frac{x^5}{5!}-{}
\nonumber\\
&& {}-\frac{x^7}{7!}+{}\cdots
\end{eqnarray}}

\begin{eqnarray}
\lefteqn{ \cos x = 1-\frac{x^2}{2!} + {}}
\nonumber\\
& & {}+\frac{x^4}{4!} - \frac{x^6}{6!}+{}\cdots
\end{eqnarray}

\subsection{phantom}
\begin{displaymath}
{}^{12}_{\phantom{1}6}\textrm{C}\qquad
\textrm{verse} \qquad
{}^{12}_{6}\textrm{C}
\end{displaymath}

\begin{displaymath}
\Gamma_{ij}^{\phantom{ij}k} \qquad
\textrm{verse} \qquad
\Gamma_{ij}^{k}
\end{displaymath}

\subsection{Size}
\begin{equation}
2^{\textrm{nd}}\qquad
2^{\mathrm{nd}}
\end{equation}

\begin{displaymath}
	\frac{\displaystyle \sum_{i=1}^{n}(x_i-\overline{x})(y_i - \overline{y})}{\displaystyle\biggl[\sum_{i=1}^{n}(x_i-\overline{x})^2\sum_{i=1}^{n}(y_i-\overline{y})^2\biggr]^{1/2}}
\end{displaymath}

\subsection{theme}

\begin{itemize}
\item[-] Usage\\
\end{itemize}
\begin{verbatim}
\newtheorem{name}[counter]{text}[section]
\end{verbatim}

%\newtheorem{theam1}{this is my interesting theorem}

\theoremstyle{definition}
\newtheorem{law}{Law}
\theoremstyle{plain}
\newtheorem{jury}[law]{Jury}
\theoremstyle{remark}
\newtheorem*{marge}{Margaret}

\begin{law}\label{law:box}
Don't hide in the witness box
\end{law}
\begin{jury}[The Twelve]
It could be you! So beware and see low~\ref{law:box}
\end{jury}
\begin{marge}
No, No, No
\end{marge}

\flushleft
\newtheorem{mur}{Murphy}[subsection]
\begin{mur}
If there are two or more ways to do something, and one of those ways can result in a catastrophe, then someone will do it.
\end{mur}

\begin{proof}
Trivial, use
\[E=mc^2\]
\end{proof}

\subsection{bold body}
\begin{displaymath}
\mu, M \qquad \mathbf{M} \qquad
\mbox{\boldmath $\mu, M$}
\end{displaymath}

\begin{displaymath}
\mu, M \qquad
\boldsymbol{\mu}, \boldsymbol{M}
\end{displaymath}
$\hbar$



\appendix
\section{appendix}
Please, start a new line right here ! \newline
Thankyou.

\end{document}